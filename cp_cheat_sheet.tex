\documentclass{article}
%\usepackage[landscape]{geometry}
\usepackage[a4paper, total={5in, 8.3in}]{geometry}
\usepackage{url}
\usepackage{multicol}
\usepackage{amsmath}
%\usepackage{esint} % removed because shows \int as \Delta
% https://stackoverflow.com/questions/68691353/in-latex-the-integral-symbol-int-displays-as-delta-in-the-compiled-pdf-w
\usepackage{amsfonts}
\usepackage{tikz}
\usetikzlibrary{decorations.pathmorphing}
\usepackage{amsmath,amssymb}

\usepackage{colortbl}
\usepackage{xcolor}
\usepackage{mathtools}
\usepackage{amsmath,amssymb}
\usepackage{enumitem}
%\makeatletter

%\newcommand*\bigcdot{\mathpalette\bigcdot@{.5}}
%\newcommand*\bigcdot@[2]{\mathbin{\vcenter{\hbox{\scalebox{#2}{$\m@th#1\bullet$}}}}}
%\makeatother

\title{Probability Final Exam Cheat Sheet}
\usepackage[utf8]{inputenc}

\advance\topmargin-.8in
\advance\textheight3in
\advance\textwidth3in
\advance\oddsidemargin-1.5in
\advance\evensidemargin-1.5in
\parindent0pt
\parskip2pt
\newcommand{\hr}{\centerline{\rule{3.5in}{1pt}}}
%\colorbox[HTML]{e4e4e4}{\makebox[\textwidth-2\fboxsep][l]{texto}
\begin{document}

\begin{center}{\huge{\textbf{Probability Final Exam Cheat Sheet}}}\\
\end{center}
\begin{multicols*}{3}

\tikzstyle{mybox} = [draw=black, fill=white, very thick,
    rectangle, rounded corners, inner sep=10pt, inner ysep=10pt]
\tikzstyle{fancytitle} =[fill=black, text=white, font=\bfseries]

%------------ Gamma function---------------
\begin{tikzpicture}
\node [mybox] (box){%
    \begin{minipage}{0.3\textwidth}
    $\Gamma(z) = \int^{\infty}_{0}t^{z - 1} e^{-t} dt$; for $\mathcal{R}(z) > 0$ \\
    $\Gamma(z + 1) = z \Gamma(z)$; $\Gamma(n + 1) = n!$ for $n \in \mathbb{N}$
    \end{minipage}
};
%------------ Gamma function Header ---------------------
\node[fancytitle, right=10pt] at (box.north west) {Gamma function};
\end{tikzpicture}

%------------ Gamma distribution ---------------
\begin{tikzpicture}
\node [mybox] (box){%
    \begin{minipage}{0.3\textwidth}
    $f_Y(y) = \frac{\beta^{\alpha}}{\Gamma(\alpha)}y^{\alpha - 1}e^{-\beta y}$; $y, \alpha, \beta > 0$ \\
    $ E[Y] = \alpha / \beta$; $Var(Y) = \alpha / \beta^{2}$ \\
    MGF$(t)= \left( 1 - \frac{t}{\beta} \right)^{-\alpha}$ for $t < \beta$ \\
    $\chi^2 (k) \sim \text{Gamma}(k/2, 1/2)$ \\
    $Y_{i}$ indip. $\text{Gamma}(\alpha_{i}, \beta)$ $\Rightarrow$ $\sum^{n}_{i = 1}Y_{i} \sim \text{Gamma}(\sum^{n}_{i = 1}\alpha_{i}, \beta)$
    \end{minipage}
};
%------------ Gamma distribution Header ---------------------
\node[fancytitle, right=10pt] at (box.north west) {Gamma distribution};
\end{tikzpicture}

%------------ Beta distribution---------------
\begin{tikzpicture}
\node [mybox] (box){%
    \begin{minipage}{0.3\textwidth}
    $f_Y(y) = \frac{\Gamma(\alpha + \beta)}{\Gamma(\alpha)\Gamma(\beta)} y^{\alpha - 1} (1 - y)^{\beta - 1}$; $\alpha,\beta > 0$, $y \in [0,1]$ \\
    $E[Y] = \frac{\alpha}{\alpha + \beta}$; $Var(Y) = \frac{\alpha \beta}{(\alpha + \beta)^2 (\alpha + \beta + 1)}$
    \end{minipage}
};
%------------ Beta distribution- Header ---------------------
\node[fancytitle, right=10pt] at (box.north west) {Beta distribution};
\end{tikzpicture}

%------------ Normal distribution ---------------
\begin{tikzpicture}
\node [mybox] (box){%
    \begin{minipage}{0.3\textwidth}
    $f_Y(y) = \frac{\det(A)^{1/2} e^{-\frac{1}{2}b^T A b}}{(2\pi)^{d/2}} \exp\{-\frac{1}{2}(y^T A y) + b^T y\}$ \\
    $E[Y] = A^{-1}b$; MGF$(t)= \exp\{\mu t + \sigma^2 t^2 / 2 \}$ \\
    $Y_1|Y_2 = y_2 \sim N(\mu_1 + \Sigma_{12}\Sigma^{-1}_{22}(y_2 - \mu_2); \Sigma_{11} - \Sigma_{12}\Sigma^{-1}_{22}\Sigma_{21})$
    \end{minipage}
};
%------------ Normal distribution Header ---------------------
\node[fancytitle, right=10pt] at (box.north west) {Multivariate Normal distribution};
\end{tikzpicture}


%------------ t-Student distribution ---------------
\begin{tikzpicture}
\node [mybox] (box){%
    \begin{minipage}{0.3\textwidth}
    $f_Y(y) = \frac{\Gamma(\frac{\nu + 1}{2})}{\sqrt{\nu \pi} \Gamma (\frac{\nu}{2})} \left(1 + \frac{y^2}{\nu} \right)^{-(\nu + 1)/2}$\\
    $\nu > 0$
    \end{minipage}
};
%------------ t-Student distribution Header ---------------------
\node[fancytitle, right=10pt] at (box.north west) {t-Student distribution};
\end{tikzpicture}

%------------ Uniform distribution topic ---------------
\begin{tikzpicture}
\node [mybox] (box){%
    \begin{minipage}{0.3\textwidth}
    $Y \sim Unif(a,b)$ \\
    $f_Y(y) = \frac{1}{b -a}$ for $y \in [a,b]$ otherwise 0\\
    $F_Y(y) = \frac{y - a}{b - a}$ for $y \in [a,b]$ \\
    $E[Y] = \frac{1}{2}(a + b)$; $Var(Y) = \frac{1}{12}(b - a)^2$
    \end{minipage}
};
%------------ Uniform distribution Header ---------------------
\node[fancytitle, right=10pt] at (box.north west) {Uniform distribution};
\end{tikzpicture}

%------------ Hypergeometric distribution Content ---------------
\begin{tikzpicture}
\node [mybox] (box){%
    \begin{minipage}{0.3\textwidth}
    $f_Y(y) = \frac{\binom{K}{y}\binom{N - K}{n - y}}{\binom{N}{n}}$\\
    $E[Y] = n \frac{K}{N}$; $Var(Y) = n \frac{N - K}{N} \frac{K}{N} \frac{N - n}{N - 1}$
    \end{minipage}
};
%------------ Hypergeometric distribution Header ---------------------
\node[fancytitle, right=10pt] at (box.north west) {Hypergeometric distribution};
\end{tikzpicture}
%------------ Negative Binomial Content ---------------------
\begin{tikzpicture}
\node [mybox] (box){%
    \begin{minipage}{0.3\textwidth}
    $f_Y(y) = \binom{k + r - 1}{k} (1 - p)^{k} p^{r}$ \\
    $E[Y] = \frac{r(1 - p)}{p}$; $Var(Y) = \frac{r(1 - p)}{p^2}$
    \end{minipage}
};
%------------ Negative Binomial Header ---------------------
\node[fancytitle, right=10pt] at (box.north west) {Negative Binomial distribution};
\end{tikzpicture}


%------------ Series Content ---------------
\begin{tikzpicture}
\node [mybox] (box){%
    \begin{minipage}{0.3\textwidth}
    $\sum^{n}_{i = 0}r^{i} = \frac{1 - r^{n+1}}{1 - r}$ for $|r| < 1$ \\
        $ e^x = \sum_{n=0}^{\infty} x^n/{n!} = $ $\lim_{n \rightarrow \infty} \left(1 + \frac{x}{n} \right)^n$


    \end{minipage}
};
%------------ Series Header ---------------------
\node[fancytitle, right=10pt] at (box.north west) {Series};
\end{tikzpicture}



%------------ Trigonometric summations Content ---------------
\begin{tikzpicture}
\node [mybox] (box){%
    \begin{minipage}{0.3\textwidth}
    $\sin(x \pm y) = \sin(x)\cos(y) \pm \sin(y)\cos(x)$ \\
    $\cos(x \pm y) = \cos(x)\cos(y) \mp \sin(x)\sin(y)$ \\
    $z^2 = x^2 + y^2 - 2xy\cos(\text{angle}(x,y))$
    \end{minipage}
};
%------------ Trigonometric summations Header ---------------------
\node[fancytitle, right=10pt] at (box.north west) {Trigonometric summations};
\end{tikzpicture}



%------------ Log base change Content ---------------
\begin{tikzpicture}
\node [mybox] (box){%
    \begin{minipage}{0.3\textwidth}
    $\log_{a}(x) = \frac{\log_{b}(x)}{\log_{b}(x)}$
    \end{minipage}
};
%------------ Log base change Header ---------------------
\node[fancytitle, right=10pt] at (box.north west) {Log base change};
\end{tikzpicture}


%------------ Derivatives topic ---------------
\begin{tikzpicture}
\node [mybox] (box){%
    \begin{minipage}{0.3\textwidth}
        $\frac{d}{dx} \sin(x) = \cos(x)$;   $\frac{d}{dx} \cos(x) = -\sin(x)$ \\
        $\frac{d}{dx} \arcsin(x) = \frac{1}{\sqrt{1 - x^2}}$;
        $\frac{d}{dx} \arccos(x) = - \frac{1}{\sqrt{1 - x^2}}$ \\
        $\frac{d}{dx} \tan(x) = \frac{1}{\cos^{2}(x)}$; $\frac{d}{dx} \arctan(x) = \frac{1}{1 + x^2}$


    \end{minipage}
};
%------------ Derivatives Header ---------------------
\node[fancytitle, right=10pt] at (box.north west) {Derivatives};
\end{tikzpicture}


%------------ Integrals topic ---------------
\begin{tikzpicture}
\node [mybox] (box){%
    \begin{minipage}{0.3\textwidth}
    $\int x^k dx = \frac{x^{k + 1}}{k + 1} + C$ $k \neq 1$; $\int \frac{1}{x} dx = \log(|x|) + C$ \\
    $\int \tan(x) dx = \log(\frac{1}{\cos(x)}) + C$ \\
    $\int \frac{1}{a^2 + x^2} dx = \frac{1}{a} \arctan(\frac{x}{a}) + C$ \\
    $\int \frac{1}{\sqrt{a^2 + x^2}} dx = \arcsin(\frac{x}{a}) + C$

    \end{minipage}
};
%------------ Integrals Header ---------------------
\node[fancytitle, right=10pt] at (box.north west) {Integrals};
\end{tikzpicture}




%------------ Taylor expansions ---------------
\begin{tikzpicture}
\node [mybox] (box){%
    \begin{minipage}{0.3\textwidth}
    $e^x = \sum^{k}_{i = 0} \frac{x^i}{i!} + o(x^k)$ for $x \rightarrow 0$ \\
    $\log(1 + x) = \sum^{k}_{i = 1} (-1)^{i - 1} \frac{x^{i}}{i} + o(x^k)$ for $x \rightarrow 0$ \\
    $\sin(x) = \sum^{n}_{i = 0}\frac{(-1)^i x^{2i + 1}}{(2i + 1)!} + o(x^{2n})$ \\ $\cos(x) = \sum^{n}_{i = 0}\frac{(-1)^i x^{2i}}{(2i)!} + o(x^{2n})$

    \end{minipage}
};
%------------ Taylor expansions Header ---------------------
\node[fancytitle, right=10pt] at (box.north west) {Taylor expansions};
\end{tikzpicture}


%------------ Function of random variable topic ---------------
\begin{tikzpicture}
\node [mybox] (box){%
    \begin{minipage}{0.3\textwidth}
    $X,Y \in \mathbb{R}^d$; $Y = g(X)$ invertible and with $\det(J_{g^{-1}}(y)) \neq 0$ then $f_{Y}(y) = f_{X}(g^{-1}(y))|\det(J_{g^{-1}}(y))|$
    \end{minipage}
};
%------------ Function of random variable Header ---------------------
\node[fancytitle, right=10pt] at (box.north west) {Function of a random variable};
\end{tikzpicture}


%------------ Jacobian topic ---------------
\begin{tikzpicture}
\node [mybox] (box){%
    \begin{minipage}{0.3\textwidth}
    $a,b \in \mathbb{R}^d$; $b = f(a)$. \\
    $\left(J_{f}(b)\right)_{ij} = \frac{df_{i}(a)}{da_{j}}$
    \end{minipage}
};
%------------ Jacobian Header ---------------------
\node[fancytitle, right=10pt] at (box.north west) {Jacobian};
\end{tikzpicture}


%------------ Matrix inversion topic ---------------
\begin{tikzpicture}
\node [mybox] (box){%
    \begin{minipage}{0.3\textwidth}
    $A^{-1} = (adj A) / \det(A)$; \\
    $$
    A = \begin{pmatrix}
    a_{11} & a_{12} & a_{13} \\
    a_{21} & a_{22} & a_{23} \\
    a_{31} & a_{32} & a_{33}
    \end{pmatrix} $$

    $$ m_{11} = \begin{pmatrix}
    a_{22} & a_{23} \\
    a_{32} & a_{33} \\
    \end{pmatrix} $$
    $$c_{i1} = (-1)^{1+i} \det(m_{i1})$$
    $$
    \det A = \sum^{n}_{i = 1} c_{i1}
    $$

    $$ C = \begin{pmatrix}
    c_{11} & c_{12} & c_{13} \\
    c_{21} & c_{22} & c_{23} \\
    c_{31} & c_{32} & c_{33}
    \end{pmatrix} $$

    $$ adj(A) = C^T $$

    \end{minipage}
};
%------------ Matrix inversion Header ---------------------
\node[fancytitle, right=10pt] at (box.north west) {Matrix inversion};
\end{tikzpicture}



%------------ Another topic ---------------
\begin{tikzpicture}
\node [mybox] (box){%
    \begin{minipage}{0.3\textwidth}
    some stuff
    \end{minipage}
};
%------------ Another topic Header ---------------------
\node[fancytitle, right=10pt] at (box.north west) {Another topic};
\end{tikzpicture}


\end{multicols*}
\end{document}
